This chapter summarizes the achieved goals of this thesis and gives answer to the research questions proposed in Chapter 1. Then a discussion of AppElastic algorithms outlook over possible work can be performed in future.

\section{Achievements}
\label{sec:Achievements}
AppElastic algorithm, as a proactive threshold based horizontal scaling algorithm, provides a customer oriented approach to elasticity for cloud hosted applications. AppElastic algorithm works in collaboration with ARIMA forecast model and relies on models ability to forecast future workloads. It is designed and implemented with ElasticSim simulator based on the principle of easy testability of scaling algorithms and modeling of various components of the cloud.
\\
With the introduction of AppElastic algorithm and ElasticSim following goal where achieved:
\begin{itemize}
  \item Scalability: AppElastic algorithm which was introduced in the section~\ref{subs:AppElastic with Look-ahead}, is the actual scaling agent which horizontally scales the virtual servers to accommodate rapidly changing workloads.
  \item SLA guarantees: The implementation of AppElastic algorithm which was provided in section~\ref{sec:AppElastic Algorithm Implementation} provides SLA aware provisioning of VM's which is an important aspect for the serving consumers.
  \item Cost benefits: ElasticSim simulator which was introduced in the section~\ref{sec:Simulation design} provides various cost saving recommendations based on historical workload analysis. Once the simulator user chooses the best purchasing option. AppElastic algorithm tries to provide best performance scaling action for the given cost.
  \item Testability: ElasticSim simulator implementation provides an easy configurable system. The output generated by the simulator helps in debugging scale algorithm and provide graphic reports for readability purpose.
\end{itemize}

Based on the achievements summarized above as well as the work introduced in the Chapter 4 and Chapter 5, the answers to the research questions proposed in Section~\ref{sec:Research Questions} can be addressed as follows:
\begin{itemize}
  \item What are the parameters to consider while computing the right number of VM resources required to satisfies QoS guarantees?
  \\
  As defined in section~\ref{sec:Classification of elasticity mechanism}, AppElastic is a threshold based automated scaling algorithm. It uses total user request to the application as threshold parameter for compute right number of VM machine resources. As it was showed in section~\ref{sub:Evaluation of AppElastic Algorithm} AppElastic algorithm provide best SLA guarantees, but due it proactive provisioning feature SLA violations will occur.
  \item What are the right VM instance types to use to achieve low operational cost?
  \\
  Based on the heuristics approach employed by ElasticSim. Amazon AWS pricing options as an example, ElasticSim provides its user best possible instance purchase option based on analysis of historical workload trace. Once the historical data is used for instance planning, right number of instance types can be configured to work on actual workload to achieve cost benefits through savings.
  \item What kind of tool and technique can be used to accurately evaluate the performance and cost of the scaling algorithm without actual deployment on the clouds?
  \\
  ElasticSim which was introduced in section~\ref{sec:Simulation design} provides a configurable system to deploy and test any horizontal scaling algorithm. This kind of simulator provides researchers configurable and easy to implement feature.
  \item What are the forecasting techniques which can be used to proactively provision VM's to mitigate effects of delay in VM start and shutdown time?
  \\
  Considering the user request workload as time series data, based on background research ARIMA forecasting techniques which was introduced in ~\ref{sub:ARIMA} was used. As proved in evaluation section~\ref{sub:Forecast Accuracy}, automated ARIMA forecasting model which was used in this thesis provides as small as 1.5\% error in forecasting future workloads.
\end{itemize}
\section{Future Work}
\label{sec:Future Work}
The propose proactive auto scaling method opens up possibility of some new and challenging problems. Some of the possible extensions of the current work are discussed as bellow:
\begin{itemize}
  \item Workload: In this thesis, workload traces from Citrix Audio/Video conferencing application is used as input. Even though it proves to provide cost benefits and SLA guarantees. It has been seen how the AppElastic algorithm performs under test workload and its remain to explore the AppElastic algorithm behavior under production. Other than deploying AppElastic in production, research community has developed various syntactic workload generation tools such as LIMBO\cite{von2014limbo}\footnote{http://descartes.tools/limbo} which can be integrated with AppElastic to model complex workload patterns.
  \item Different forecasting techniques: ARIMA time series forecasting is the only modeling technique applied. Herbst et al.\cite{herbst2012workload} have developed a technique called Workload Classification \& Forecasting (WCF) tool which automatically generates spectrum of forecasting methods based on time series analysis like ARIMA, Extended Exponential Smoothing (ETS), etc\cite{herbst2012workload}\footnote{https://github.com/NikolasHerbst/WCF}. Hence integrating AppElastic with such modern tools can be one the future works.
  \item Different threshold parameters: In the current system, number of user request is used as threshold parameter for scaling the system. Performance profiling of an application will provide various other parameters such as CPU load, memory, network utilization, service response time etc. which can be included into AppElastic algorithm as further threshold parameters.
\end{itemize}
