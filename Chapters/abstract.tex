
Recently cloud computing is know to change the way applications are designed and deployed. Cloud systems use virtualization technologies (such as Xen, KVM etc.) to allocate on-demand IT resources in terms of virtual machines (VM). This on-demand allocation is called scaling. Scalability is a critical issue to the success of application involved in using cloud as hosting platform. Organizations which use cloud for hosting their application have Quality of Service (QoS) guaranties to fulfill in accordance with service level agreements with their customers. Automated scaling mechanisms hold the promise of assuring QoS properties to the applications while simultaneously making efficient use of resources and keeping operational cost low.
\\
\\
Despite the advantages of automated scaling, realizing the full potential of automated scaling is hard due to the challenge of precisely estimating the resource need when system has arbitrary workload. Along with arbitrary workload, Infrastructure-As-A-Service (IaaS) cloud providers (such as Amazon AWS) offer diverse virtual machine instance purchasing options, such as on-demand instances, reserved instances, and spot instances. Such diverse pricing models make it challenging to determine how to optimally provision the required number of VM instances in different types to satisfy arbitrary workload demands. Given the limited budget, service providers need to carefully balance the VM provisioning cost and achieve QoS for end users.
\\
\\
This thesis introduces a threshold based automated scaling algorithm, named AppElastic. AppElastic algorithm deals with the horizontal scalability in IaaS mode. It address the trade-off problem between cost and QoS guaranties by employing very popular threshold based rules to automate scaling. AppElastic employs time series forecasting to proactively provision VM needed to satisfy the QoS guaranties. Furthermore, to mitigate rapid starting and stopping of VM's due to arbitrary workload, AppElastic algorithm is developed to terminate the VM's which are nearing instance-hour. The developed system is a customer oriented solution that could be extended to any cloud application to achieve horizontal scalability.
