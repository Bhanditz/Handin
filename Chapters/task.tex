\paragraph{Objective}
\label{par:Objective}
Elastic and adaptive auto-scaling are foundation capabilities of application bound to on-demand infrastructure services in the cloud. The services offer heterogeneous characteristics (e.g price, capacity, locality), which need to mach the application needs at any point in time. For real-time communication applications, it is utmost important to keep sufficient resources available to ensure a high quality of service for all customers, whose services level expectations are highly influenced by the type of communication (audio, video, screen sharing). At the same time, the cost caused by idle resources should be reduced. The SPIDAR project already delivers an advisory service, which turns the estimation of service users and adjustments into rudimentary scaling decisions.\\
The goal of this work is the conception of more sophisticated Resource Planning Component (RPC) that is responsible for creating so-called action plans. Action plans contain information when to scale up/down which compute resource to fulfill  all service levels with minimum infrastructure costs. The Resource Planning Component should take user, communication and infrastructure characteristics into consideration to provide action plans that fulfill the requirements of service quality and costs for a combined and elastic audio/video communication service infrastructure.\\

\paragraph{MAIN FOCUS}
\label{par:MAIN FOCUS}
\begin{itemize}
  \item Analysis of Existing methods for scaling cloud environments in terms of service level agreements and infrastructural costs
  \item Design of a Resource Planning Component for generating action plans based on various parameters (e.g. user type, service levels, infrastructure costs, billing period)
  \item Implementation and intergation into the SPIDAR prototype
  \item Evaluation of the Resource Planning Component based on simulations.
\end{itemize}
